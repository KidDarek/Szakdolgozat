\chapter{Bevezetés}


A játékok világa egy hatalmas és sokmindent magábafoglaló tér. Legyen szó egy egyszerűbb platformer játékról mint a Mario vagy pedig egy többszintű hatalmas és komplex CRPG-ről mint a Baldur's Gate 3. A játékok egy sokszinű világot nyitnak meg a játékosok számára ahol új élményeket, történeteket és kalandokat élhetnek át. A készítők számára egy kreatív fórumot ad a gondolataik, élményeik és víziójik megvalósítására.

Ezen világon belül foglalnak el helyet a kártya játékok. A kártyajáték mint megnevezés sokmindent jelenthet. Gondolhatunk a tradícionális kártyajátékokra mint a Póker vagy Blackjack, de ugyanakkor ide tartoznak a Trading Card Game-ek is például Magic The Gathering vagy a Yu-Gi-Oh! . Stílustól függetlenül minden kártyajátékban központi szerepet tölt be a :
\begin{itemize}
    \item Pakli amit a játékos használ 
    \item A kártyák a pakliban. Ezek lehetnek előre meghatározottak vagy a játékos által választottak
    \item A játék szabályrendszere. Ez határozza meg hogy a játék hogyan működik.
\end{itemize}

Nehézség illetve előny ez a kevés központi pillér amire ezek a játékok építkeznek. Nagy szabadságot ad a készítőnek de ez a fajta kötetlenség megnehezítheti a konkrét program elkészítését. 

Szakdolgozatomban ezen alapgondolatokra építve készítettem el egy egyszemélyes tradícionális RPG karakter rendszerre alapuló kártyajátékot. A játék fő fúkusza a karakterek klassok  sajátos játékstílusa, a pakli építése , klassok kártyái közötti keverés lehetősége illetve egy számítógépi ellenfél ami ellen a játékos tud játszani. A játék 8 karakter klasst tartalmaz amelyek mind más játékstílust reprezentálnak és más erősségel rendelkeznek. Minden klass 5 klass kártyával rendelkezik. Ezek a kártyák a karakter játékstílusához igazodnak. 

